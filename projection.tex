% (C) 2022 Yung-Yu Chen.  All rights reserved.
% chktex-file 7

\RequirePackage[2020-02-02]{latexrelease}
\documentclass[11pt,dvips]{article}
%\documentclass[12pt,dvips]{article}
%\documentclass[preprint,dvips,numbers,sort&compress]{elsarticle}
%\documentclass[12pt,review,dvips,numbers,sort&compress]{elsarticle}
% Geometry.
\usepackage{geometry}
\geometry{a4paper,
left=2cm,
right=2cm,
top=2cm,
bottom=2cm,
}
% Global functionalities.
\usepackage[numbers,sort&compress]{natbib}
\usepackage{hyperref}
% encoding.
\usepackage[utf8]{inputenc}
% Mathematics.
\usepackage{amsmath}
\usepackage{amssymb}
\usepackage{amsfonts}
\usepackage{amsthm}
\usepackage{arydshln}
% Authoring.
%\usepackage{authblk}
\usepackage[printwatermark]{xwatermark}
\usepackage{xcolor}
\usepackage{graphicx}
\usepackage{subfigure}
\usepackage{paralist}
\usepackage{multirow}
\usepackage{booktabs}
\usepackage{setspace}
%\usepackage{lmodern}

\newwatermark[allpages,color=black!30,angle=45,scale=3,xpos=0,ypos=0]%
{DRAFT}

%\doublespacing

\graphicspath{{turgon_eps/}}

\renewcommand{\figurename}{Fig.}
\newcommand{\topcaption}{%
\setlength{\abovecaptionskip}{0pt}%
\setlength{\belowcaptionskip}{10pt}%
\caption}
\numberwithin{equation}{section}

\newcommand{\defeq}{\ensuremath{\buildrel {\text{def}}\over{=}}}

\title{
%
(Drafting) Projection Method
%
}

\author{
%
Yi-En Chou and Yung-Yu Chen
%
}

\begin{document}

\maketitle

\begin{abstract}
%
%
\end{abstract}


\section{Governing equations for 2D incompressible flow}\label{s:projection}

Flow characteristics of viscous incompressible flow is depicted via the law of 
conservation of momentum, for fluid it is the Navier-Stokes equation, and the 
law of conservation of mass, or the continuity. Suppose $\rho=1$ for simplicity, 
momentum equation in $x_1$-direction:
%
\begin{align}
\frac{\partial v_1}{\partial t}+v_1 \frac{\partial v_1}{\partial x_1}+v_2 \frac{\partial v_1}
{\partial x_2}=-\frac{\partial p}{\partial x_1}+ \nu (\frac{\partial^2 v_1}{\partial x_1^2}
+\frac{\partial^2 v_1}{\partial x_2^2}) \label{e:x_momentum}
\end{align}
%
Momentum equation in $x_2$-direction:
%
\begin{align}
\frac{\partial v_2}{\partial t}+v_1 \frac{\partial v_2}{\partial x_1}+v_2 \frac{\partial v_2}
{\partial x_2}=-\frac{\partial p}{\partial x_2}+ \nu (\frac{\partial^2 v_2}{\partial x_1^2}
+\frac{\partial^2 v_2}{\partial x_2^2}) \label{e:y_momentum}
\end{align}
%
Continuity:  
%
\begin{align}
\frac{\partial v_1}{\partial x_1}+\frac{\partial v_2}{\partial x_2}=0 \label{e:continuity}
\end{align}
%
\subsection{Vector form}
Governing equations can be written in vector form, where the momentum equations 
in vector form is:
%
\begin{align}
\mathbf{v}_t+\mathbf{v}(\nabla \cdot \mathbf{v})=-\nabla p+\nu \nabla^2 \mathbf{v} 
\label{e:momentum_vec}
\end{align}
%
Continuity:
%
\begin{align}
\nabla \cdot \mathbf{v}=0 \label{e:continuity_vec}
\end{align}
%
\section{Projection method}
Velocity at $n^{th}$ time-step is denoted as $\mathbf{v}^n$. Convection term is 
denoted as $\mathbf{C}$, and diffusion term is denoted $\mathbf{D}$.
%
\begin{align}
\frac{\mathbf{v}^{n+1}-\mathbf{v}^{n}}{\Delta t}+\mathbf{C}(\mathbf{v}^{n})=
-\nabla p+\mathbf{D}(\mathbf{v}^{{n}}) \label{e:momentum_pj} 
\end{align}
%
Projection method\cite{ferziger_cfd_2019} split eq (\ref{e:momentum_pj}) into
%
\begin{gather}
\frac{\mathbf{v}^{*}-\mathbf{v}^{n}}{\Delta t}+\mathbf{C}(\mathbf{v}^{n})=-\nabla 
p^{n}+\mathbf{D}(\mathbf{v}^{{n}}) \label{e:projection1}\ \\
\frac{\mathbf{v}^{**}-\mathbf{v}^{*}}{\Delta t}=\nabla p^n \label{e:projection2}\\
\frac{\mathbf{v}^{n+1}-\mathbf{v}^{**}}{\Delta t}=-\nabla p^{n+1}  \label{e:projection3}\
\end{gather}
%
having~eq (\ref{e:projection3}) taken divergence, the equation becomes
%
\begin{align}
\frac{\nabla \cdot(\mathbf{v}^{n+1}-\mathbf{v}^{**})}{\Delta t}=-\nabla ^2 p
\label{e:div_momentum_pj_b}
\end{align}
%
A distinguishing feature of projection method is that the velocity field is 
forced to satisfy the continuity as a part of the algorithm, where
%
\begin{align}
\nabla \cdot(\mathbf{v}^{n+1})=0 \label{e:continuity_pj}
\end{align}
%
and therefore the equation becomes, 
%
\begin{align}
\frac{\nabla \cdot \mathbf{v}^{**}}{\Delta t}=\nabla ^2 p \label{e:projection4}
\end{align}
%
\subsection{The four steps of projection algorithm}
Step 1- Solve $\mathbf{v}^{*}$ with eq (\ref{e:projection1})
%
\begin{align}
\frac{\mathbf{v}^{*}-\mathbf{v}^{n}}{\Delta t}+\mathbf{C}(\mathbf{v}^{n})=
\mathbf{D}(\mathbf{v}^{{n}})  \nonumber
\end{align}
%
Step 2- Solve $\mathbf{v}^{**}$ with eq (\ref{e:projection2})
%
\begin{align}
\frac{\mathbf{v}^{**}-\mathbf{v}^{*}}{\Delta t}=\nabla p^n \nonumber
\end{align}
%
Step 3- Solve $p$ with eq (\ref{e:projection4})
%
\begin{align}
\frac{\nabla \cdot \mathbf{v}^{**}}{\Delta t}=\nabla ^2 p \nonumber
\end{align}
%
Step 4- Solve $\mathbf{v}^{n+1}$  with eq (\ref{e:projection3})
%
\begin{align}
\frac{\mathbf{v}^{n+1}-\mathbf{v}^{**}}{\Delta t}=-\nabla p^{n+1} \nonumber
\end{align}
%
\section{Discretization}
\subsection{Taylor-Series Expansion}
Discretized equations consists of approximating derivatives with truncated Taylor
-Series Expansion\cite{patankar_numerical_1980}, where Talor-series expansion 
about position $x$ are
%
\begin{gather}
f(x+h)=f(x)+f'(x)\frac{h}{1!}+f''(x)\frac{h^2}{2!}+f^{(3)}(x)\frac{h^3}{3!}+ \ldots  \label{e:FD} \\
f(x-h)=f(x)-f'(x)\frac{h}{1!}+f''(x)\frac{h^2}{2!}-f^{(3)}(x)\frac{h^3}{3!}+ \ldots  \label{e:BD} 
\end{gather}
%
First derivatives is obtained via eq (\ref{e:FD})-eq (\ref{e:BD}) truncating the 
third term in the series, where
%
\begin{gather}
f(x+h)-f(x-h)=2\times f'(x)\frac{h}{1!}+2 \times f^{(3)}(x)\frac{h^3}{3!}+ \ldots \nonumber \\
f(x+h)-f(x-h)=2\times f'(x)\frac{h}{1!}+O(h^3) \nonumber \\
f'(x)=\frac{f(x+h)-f(x-h)}{2h}  \label{e:f_1d}
\end{gather}
%
Second derivative via eq (\ref{e:FD})+eq (\ref{e:BD}) truncating the fourth term 
in the series, where
%
\begin{gather}
f(x+h)+f(x-h)=2\times f(x)+2 \times f''(x)\frac{h^2}{2!}+4 \times f^{(4)}(x)
\frac{h^4}{4!} \nonumber \\
f(x+h)+f(x-h)=2\times f(x)+2 \times f''(x)\frac{h^2}{2!}+O(h^4) \nonumber \\
f''(x)=\frac{f(x+h)+f(x-h)-2\times f(x)}{h^2} \label{e:f_2d}
\end{gather}
%
\subsection{The discretized three steps of projection algorithm}
Step 1- Solve $\mathbf{v}^{*}$ with eq (\ref{e:projection1})
%
\begin{align}
\frac{\mathbf{v}^{*}-\mathbf{v}^{n}}{\Delta t}+\mathbf{C}(\mathbf{v}^{n})=
\mathbf{D}(\mathbf{v}^{{n}})  \nonumber
\end{align}
%
where
%
\begin{gather}
\mathbf{C}({v_1}^n_{i,j})={v_1}^n_{i,j} \cdot \frac{{v_1}^n_{e}-{v_1}^n_{w}}{\Delta x_1}+{v_2}^n_{p} 
\cdot \frac{{v_1}^n_{n}-{v_1}^n_{s}}{\Delta x_2} \\
\mathbf{D}({v_1}^n_{i,j})=\nu (\frac{{v_1}^n_{e}+ {v_1}^n_{w}-2{v_1}^n_{i,j}}{{(\Delta x_1/2)}^2}
+\frac{{v_1}^n_{n}+ {v_1}^n_{s}-2{v_1}^n_{i,j}}{{(\Delta x_2/2)}^2})
\end{gather}
%
Step 2- Solve $\mathbf{v}^{**}$ with eq (\ref{e:projection2})
%
\begin{align}
\frac{\mathbf{v}^{**}-\mathbf{v}^{*}}{\Delta t}=\nabla p^n \nonumber
\end{align}
%
where
%
\begin{gather}
\nabla \cdot \mathbf{v}^{**}=\frac{{v_1}^{**}_{e}-{v_1}^{**}_{w}}{\Delta x_1}
+\frac{{v_2}^{**}_{n}-{v_2}^{**}_{s}}{\Delta x_2} \\
\nabla \cdot \mathbf{v}^{*}=\frac{{v_1}^{*}_{e}-{v_1}^{*}_{w}}{\Delta x_1}
+\frac{{v_2}^{*}_{n}-{v_2}^{*}_{s}}{\Delta x_2} \\
\nabla p^n=\frac{p^n_{i+1,j}-p^n_{i-1,j}}{2\Delta x_1} \cdot \hat{i}+\frac{p^n_{i,j+1}
-p^n_{i,j-1}}{2\Delta x_2}  \cdot \hat{j}
\end{gather}
%
Step 3- Solve $p^{n+1}$ with eq (\ref{e:projection4})
%
\begin{align}
\frac{\nabla \cdot \mathbf{v}^{**}}{\Delta t}=\nabla ^2 p^{n+1} \nonumber
\end{align}
%
where
%
\begin{gather}
\nabla ^2 p^{n+1}=\frac{p^{n+1}_{i+1,j}+p^{n+1}_{i-1,j}-2p^{n+1}_{i,j}}{\Delta x_1^2}
+\frac{p^{n+1}_{i,j+1}+p^{n+1}_{i,j-1}-2p^{n+1}_{i,j}}{\Delta x_2^2}
\end{gather}
%
Step 4- Solve $\mathbf{v}^{n+1}$ with eq (\ref{e:projection3})
%
\begin{align}
\frac{\mathbf{v}^{n+1}_{i,j}-\mathbf{v}^{**}_{i,j}}{\Delta t}=-\nabla p \nonumber
\end{align}
%
where
%
\begin{align}
\nabla p^{n+1}=\frac{p^{n+1}_{i+1,j}-p^{n+1}_{i-1,j}}{2\Delta x_1} \cdot 
\hat{i}+\frac{p^{n+1}_{i,j+1}
-p_{i,j-1}}{2\Delta x_2}  \cdot \hat{j}
\end{align}

\clearpage

\addcontentsline{toc}{chapter}{Bibliography}
%\bibliographystyle{myunsrtnat} % no sort (order in appearance)
\bibliographystyle{myplainnat} % sort by author
\bibliography{turgon_main}

\end{document}

% vim: set ff=unix fenc=utf8 et sw=2 ts=2 tw=79:
