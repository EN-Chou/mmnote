% (C) 2022 copyright Yung-Yu Chen.  All rights reserved.

\RequirePackage[2020-02-02]{latexrelease}
\documentclass[11pt,dvips]{article}
%\documentclass[12pt,dvips]{article}
%\documentclass[preprint,dvips,numbers,sort&compress]{elsarticle}
%\documentclass[12pt,review,dvips,numbers,sort&compress]{elsarticle}
% Geometry.
\usepackage{geometry}
\geometry{a4paper,
left=2cm,
right=2cm,
top=2cm,
bottom=2cm,
}
% Global functionalities.
\usepackage[numbers,sort&compress]{natbib}
\usepackage{hyperref}
% encoding.
\usepackage[utf8]{inputenc}
% Mathematics.
\usepackage{amsmath}
\usepackage{amssymb}
\usepackage{amsfonts}
\usepackage{amsthm}
\usepackage{arydshln}
% Authoring.
%\usepackage{authblk}
\usepackage[printwatermark]{xwatermark}
\usepackage{xcolor}
\usepackage{graphicx}
\usepackage{subfigure}
\usepackage{paralist}
\usepackage{multirow}
\usepackage{booktabs}
\usepackage{setspace}
%\usepackage{lmodern}

\newwatermark[allpages,color=black!30,angle=45,scale=3,xpos=0,ypos=0]%
{DRAFT}

%\doublespacing

\graphicspath{{turgon_eps/}}

\renewcommand{\figurename}{Fig.}
\newcommand{\topcaption}{%
\setlength{\abovecaptionskip}{0pt}%
\setlength{\belowcaptionskip}{10pt}%
\caption}
\numberwithin{equation}{section}

\newcommand{\defeq}{\ensuremath{\buildrel {\text{def}}\over{=}}}

\title{
%
(Drafting) Unstructured Meshes for the Conservation Element and Solution
Element Method
%
}

\author{
%
Yung-Yu Chen
%
}

\begin{document}

\maketitle

\begin{abstract}
%
Describe how the unstructured meshes work for the conservation element and
solution element (CESE) method and SOLVCON.
%
\end{abstract}

\section{Unstructured Meshes Concepts}
%
\label{s:concept}

The conservation element and solution element (CESE) method is developed
against the set-up of unstructured meshes in multi-dimensional
space\cite{mavriplis_unstructured_1997, wang_2d_1999}.  In contrast to
structured meshes, unstructured meshes allow flexible connectivity and simplex
elements.  The implementation, i.e., the data structures and the computer code
for their algorithms, of unstructured meshes dictate how simulation software
operates.  It serves two purposes: numerical methods for simulation and mesh
generation.  SOLVCON already includes code for the former and starts to plan
for the latter.

The library for unstructured meshes offers three functionalities:
%
\begin{enumerate}
%
\item Resource management: allocation, deallocation, and tracking.
%
\item Geometrical entity creation, read, update, and deletion (CRUD).
%
\item Spatial indexing.
%
\end{enumerate}

\clearpage
%
\section{Data Store}
%
\label{s:storage}

Most operations on meshes done by the simulation are reading.  The mesh is
usually assumed to be constant.  Numerical methods may use moving meshes, but
it is an advanced topic that should be treated separately.

SOLVCON uses a set of lookup tables to store the unstructured mesh.  The
technique is commonly seen in unstructured-mesh solvers for the efficient
memory use.  The CESE method is finite-volume-based and associates variables
with volume centers.  The data store optimizes for easily reading values for
discrete volume, and thus defines \textit{cell}, \textit{face}, and
\textit{node}.  Cells are the discrete volume for the space of interest.  Faces
are the interface between two cells.  Nodes represent the coordinates in space.
A mesh is also a Voronoi diagram, and the ``cell'' is a \textit{Voronoi
cell}\cite{berg_computational_2010}.  At this point, it's not straight-forward
why the three concepts are fundamental to the CESE method.  It may be revealed
in detail discussions later.

SOLVCON allows mixing elements of different shapes.  The mesh definition data
are are listed in Table~\ref{t:elm:meta}, \ref{t:subent1d}, \ref{t:subent2d},
\ref{t:subent3d} and Figure~\ref{f:elm2d}, \ref{f:elm3d}.  The code for
unstructured meshes in SOLVCON has been organized to a C++ library
\href{https://github.com/solvcon/solvcon/tree/master/libmarch}{\texttt{libmarch}}.

%%%%%%%%%%%%%%%%%%%%%%%%%%%%%%%%%%%%%%%%%%%%%%%%%%%%%%%%%%%%%%%%%%%%%%%%%%%%%%%%
%
\begin{table}[h]
\centering

\topcaption{
%
Meta data of the unstructured-mesh elements.
%
}

\label{t:elm:meta}
\begin{tabular}{lrrrrr}
\toprule
Name          & Type & Dimension & Number of & Number of & Number of \\
              &      &           &    Points &     Lines &  Surfaces \\
\midrule
Point         & 0    & 0         & 1         & (n/a) 0   & (n/a) 0   \\
Line          & 1    & 1         & 2         & (n/a) 0   & (n/a) 0   \\
Quadrilateral & 2    & 2         & 4         & 4         & (n/a) 0   \\
Triangle      & 3    & 2         & 3         & 3         & (n/a) 0   \\
Hexahedron    & 4    & 3         & 8         & 12        & 6         \\
Tetrahedron   & 5    & 3         & 4         & 6         & 4         \\
Prism         & 6    & 3         & 6         & 9         & 5         \\
Pyramid       & 7    & 3         & 5         & 8         & 5         \\
\bottomrule
\end{tabular}
\end{table}
%
%%%%%%%%%%%%%%%%%%%%%%%%%%%%%%%%%%%%%%%%%%%%%%%%%%%%%%%%%%%%%%%%%%%%%%%%%%%%%%%%

%%%%%%%%%%%%%%%%%%%%%%%%%%%%%%%%%%%%%%%%%%%%%%%%%%%%%%%%%%%%%%%%%%%%%%%%%%%%%%%%
%
\begin{table}[h]
\centering

\topcaption{
%
The relation between a one-dimensional element and its sub-entities.
%
}

\label{t:subent1d}
\begin{tabular}{lll}
\toprule
Shape (type) & Face & = Node \\
\midrule
Line (1)     & 0    & 0      \\
             & 1    & 1      \\
\bottomrule
\end{tabular}
\end{table}
%
%%%%%%%%%%%%%%%%%%%%%%%%%%%%%%%%%%%%%%%%%%%%%%%%%%%%%%%%%%%%%%%%%%%%%%%%%%%%%%%%

%%%%%%%%%%%%%%%%%%%%%%%%%%%%%%%%%%%%%%%%%%%%%%%%%%%%%%%%%%%%%%%%%%%%%%%%%%%%%%%%
%
\begin{table}[h]
\centering

\topcaption{
%
The relations between two-dimensional elements and their sub-entities.  Both of
two-dimensional elements are enclosed by straight lines.  Node orientation of
two-dimensional elements is defined in Fig.~\ref{f:elm2d}.
%
}

\label{t:subent2d}
\begin{tabular}{lll}
\toprule
Shape (type)      & Face & = Line formed by nodes \\
\midrule
Quadrilateral (2) & 0    & $\diagup$ 0 1          \\
                  & 1    & $\diagup$ 1 2          \\
                  & 2    & $\diagup$ 2 3          \\
                  & 3    & $\diagup$ 3 0          \\
\midrule
Triangles (3)     & 0    & $\diagup$ 0 1          \\
                  & 1    & $\diagup$ 1 2          \\
                  & 2    & $\diagup$ 2 0          \\
\bottomrule
\end{tabular}
\end{table}
%
%%%%%%%%%%%%%%%%%%%%%%%%%%%%%%%%%%%%%%%%%%%%%%%%%%%%%%%%%%%%%%%%%%%%%%%%%%%%%%%%

%%%%%%%%%%%%%%%%%%%%%%%%%%%%%%%%%%%%%%%%%%%%%%%%%%%%%%%%%%%%%%%%%%%%%%%%%%%%%%%%
%
\begin{table}[h]
\centering

\topcaption{
%
The relations between three-dimensional elements and their sub-entities.
Three-dimensional elements are enclosed by triangles or quadrilaterals, or a
combination of them.  $\square$ in the third column denotes quadrilaterals,
while $\triangle$ triangles.  Nodes in the third column are ordered so that the
normal vector of a surface points outward from the volume by following
right-hand rule.  Node orientation of three-dimensional elements is defined in
Fig.~\ref{f:elm3d}.
%
}

\label{t:subent3d}
\begin{tabular}{lll}
\toprule
Shape (type)    & Face & = Surface formed by nodes \\
\midrule
Hexahedron (4)  & 0    & $\square$ 0 3 2 1         \\
                & 1    & $\square$ 1 2 6 5         \\
                & 2    & $\square$ 4 5 6 7         \\
                & 3    & $\square$ 0 4 7 3         \\
                & 4    & $\square$ 0 1 5 4         \\
                & 5    & $\square$ 2 3 7 6         \\
\midrule
Tetrahedron (5) & 0    & $\triangle$ 0 2 1         \\
                & 1    & $\triangle$ 0 1 3         \\
                & 2    & $\triangle$ 0 3 2         \\
                & 3    & $\triangle$ 1 2 3         \\
\midrule
Prism (6)       & 0    & $\triangle$ 0 1 2         \\
                & 1    & $\triangle$ 3 5 4         \\
                & 2    & $\square$ 0 3 4 1         \\
                & 3    & $\square$ 0 2 5 3         \\
                & 4    & $\square$ 1 4 5 2         \\
\midrule
Pyramid (7)     & 0    & $\triangle$ 0 4 3         \\
                & 1    & $\triangle$ 1 4 0         \\
                & 2    & $\triangle$ 2 4 1         \\
                & 3    & $\triangle$ 3 4 2         \\
                & 4    & $\square$ 0 3 2 1         \\
\bottomrule
\end{tabular}
\end{table}
%
%%%%%%%%%%%%%%%%%%%%%%%%%%%%%%%%%%%%%%%%%%%%%%%%%%%%%%%%%%%%%%%%%%%%%%%%%%%%%%%%

%%%%%%%%%%%%%%%%%%%%%%%%%%%%%%%%%%%%%%%%%%%%%%%%%%%%%%%%%%%%%%%%%%%%%%%%%%%%%%%%
%
\begin{figure}[h]
\centering
\subfigure[]{
  \includegraphics{elm_ln.eps}
  \label{f:elm2d:ln}
}
\subfigure[]{
  \includegraphics{elm_quad.eps}
  \label{f:elm2d:quad}
}
\subfigure[]{
  \includegraphics{elm_tri.eps}
  \label{f:elm2d:tri}
}

\caption{
%
Node definition of one- and two-dimensional mesh elements in SOLVCON:
%
\subref{f:elm2d:ln} Line (type number 1).
%
\subref{f:elm2d:quad} Quadrilateral (type number 2).
%
\subref{f:elm2d:tri} Triangle (type number 3).
%
}

\label{f:elm2d}
\end{figure}
%
%%%%%%%%%%%%%%%%%%%%%%%%%%%%%%%%%%%%%%%%%%%%%%%%%%%%%%%%%%%%%%%%%%%%%%%%%%%%%%%%

%%%%%%%%%%%%%%%%%%%%%%%%%%%%%%%%%%%%%%%%%%%%%%%%%%%%%%%%%%%%%%%%%%%%%%%%%%%%%%%%
%
\begin{figure}[h]
\centering
\subfigure[]{
  \includegraphics{elm_hex.eps}
  \label{f:elm3d:hex}
}
\subfigure[]{
  \includegraphics{elm_tet.eps}
  \label{f:elm3d:tet}
}
\subfigure[]{
  \includegraphics{elm_psm.eps}
  \label{f:elm3d:psm}
}
\subfigure[]{
  \includegraphics{elm_pym.eps}
  \label{f:elm3d:pym}
}

\caption{
%
Node definitions of three-dimensional mesh elements in SOLVCON:
%
\subref{f:elm3d:hex} Hexahedron (type number 4).
%
\subref{f:elm3d:tet} Tetrahedron (type number 5).
%
\subref{f:elm3d:psm} Prism (type number 6).
%
\subref{f:elm3d:pym} Pyramid (type number 7).
%
}

\label{f:elm3d}
\end{figure}
%
%%%%%%%%%%%%%%%%%%%%%%%%%%%%%%%%%%%%%%%%%%%%%%%%%%%%%%%%%%%%%%%%%%%%%%%%%%%%%%%%

\clearpage
\section{The CESE Method Dual Mesh}
\label{s:cese_dual}

Figure~\ref{f:mesh_2d_tri} exhibits 6 triangles as an example of mesh elements
for the CESE method.  The CESE method evaluates the solutions at the centroids
of conservation elements (CEs).  The centroids are the solution points and are
used to construct the solution elements (SEs).  The element centers and the
mesh vertices consist of the conservation elements.  The conservation element
is the space-time dual mesh defined on the unstructured mesh for the CESE
method.  See Figure~\ref{f:mesh_2d_ce}.

%%%%%%%%%%%%%%%%%%%%%%%%%%%%%%%%%%%%%%%%%%%%%%%%%%%%%%%%%%%%%%%%%%%%%%%%%%%%%%%%
%
\begin{figure}[h]
\centering
\includegraphics{mesh_2d_tri.eps}
\caption{Triangular mesh in two-dimensional space.}
\label{f:mesh_2d_tri}
\end{figure}

\begin{figure}[h]
\centering
\includegraphics{mesh_2d_ce.eps}
\caption{Conservation elements of triangular meshes.}
\label{f:mesh_2d_ce}
\end{figure}
%
%%%%%%%%%%%%%%%%%%%%%%%%%%%%%%%%%%%%%%%%%%%%%%%%%%%%%%%%%%%%%%%%%%%%%%%%%%%%%%%%

\clearpage

The CESE method $c$-scheme composes of evaluations of conservation and
gradients.  The first part assumes the gradients of the previous half time step
as known to calculate the primary variables.  The second part calculates the
first-order derivative.  To calculate the first-order derivative, we define
\textit{gradient elements} (GEs)\cite{chen_multi-physics_2011}.  There are two
types of GEs: \textit{fundamental GE} (FGE) and \textit{generalized GE} (GGE).
A FGE is an simplex in $\mathbb{E}^N$ space.  It always has $N+1$ vertices.  A
GGE is a convex element composes of multiple non-overlapping FGEs that are
separated by the GGE centroid.

In a FGE, the gradient of a scalar function $\phi(\mathbf{x})$ is assumed to be
constant, and denoted by
\begin{align}
  \mathbf{g} \defeq \nabla\phi \label{e:fge:grad}
\end{align}
Let $\mathbf{x}^{(i)}$, $i = 0, 1, \ldots, N$ be the coordinates of the
vertices of a FGE.  The coordinates define $N$ \textit{displacement vectors}
\begin{align}
  \mathbf{d}^{(i)} \defeq \mathbf{x}^{(i)} - \mathbf{x}^{(0)},
  \quad i = 1, \ldots, N \label{e:fge:dis_vec}
\end{align}
Combine all the displacement vectors to write the \textit{displacement matrix}
\begin{align}
  D \defeq \left(\begin{array}{ccc}
    d^{(1)}_1 & \hdots & d^{(1)}_N \\
    \vdots & \ddots & \vdots \\
    d^{(N)}_1 & \hdots & d^{(N)}_N
  \end{array}\right) \label{e:fge:dis_mat}
\end{align}
Define
\begin{align}
  \mathbf{q} \defeq \left(\begin{array}{c}
    \phi(\mathbf{x}^{(1)}) - \phi(\mathbf{x}^{(0)}) \\
    \vdots \\
    \phi(\mathbf{x}^{(N)}) - \phi(\mathbf{x}^{(0)})
  \end{array}\right) \label{e:fge:dif_vec}
\end{align}
and call it the \textit{difference vector}.  The system equation $\mathbf{q} =
\mathrm{D}\mathbf{g}$ can be written.  $\mathbf{q}$ and $\mathrm{D}$ are known
and $\mathbf{g}$ is unknown.  Write
\begin{align}
  \mathbf{g} = \mathrm{D}^{-1}\mathbf{q} \label{e:fge:solve_grad}
\end{align}
The gradient $\mathbf{g}$ defined in Eq.~(\ref{e:fge:grad}) is determined by
Eqs~(\ref{e:fge:dis_vec}), (\ref{e:fge:dis_mat}), (\ref{e:fge:dif_vec}), and
(\ref{e:fge:solve_grad}).

The gradient of a GGE is approximated by the average gradient at its centroid
\begin{align}
  \mathbf{g}^c \defeq \frac{1}{M} \sum_{i=0}^{M-1} \mathbf{g}^{(i)}
  \label{e:gge:grad:centroid}
\end{align}
where $\mathbf{g}^{(0)}, \mathbf{g}^{(1)}, \ldots, \mathbf{g}^{(M-1)}$ are the
gradient of its FGEs.  If the GGE is a simplex, i.e., $M = N+1$, it can be
shown that $\mathbf{g}^c$ is equal to the gradient calculated by treating the
GGE as a FGE and applying Eq.~\ref{e:fge:solve_grad}.

Equation \ref{e:gge:grad:centroid} leads to more interesting weighting
functions for approximating GGE gradient for treating discontinuity.  The W-1
scheme uses
\begin{align}
  \mathbf{g}^{c} \approx \dfrac{
    \sum\limits_{i=0}^{M-1}\left(\rho^{(i)}\right)^{\alpha}\mathbf{g}^{(i)}
  }{
    \sum\limits_{i=0}^{M-1}\left(\rho^{(i)}\right)^{\alpha}
  }
\end{align}
where the weighting coefficients
\begin{align}
  \rho^{(i)} \defeq \prod_{k=0; k\neq i}^{M-1}
                    \left\vert\mathbf{g}^{(k)}\right\vert,
  \quad i = 0, \ldots, M-1
\end{align}
and $\alpha$ an adjustable parameter, usually a natural number.

\clearpage

\addcontentsline{toc}{chapter}{Bibliography}
%\bibliographystyle{myunsrtnat} % no sort (order in appearance)
\bibliographystyle{myplainnat} % sort by author
\bibliography{turgon_main}

\end{document}

% vim: set ai et sw=2 ts=2 tw=79:
